% This is samplepaper.tex, a sample chapter demonstrating the
% LLNCS macro package for Springer Computer Science proceedings;
% Version 2.20 of 2017/10/04
%
\documentclass[runningheads]{llncs}
%
\usepackage[utf8]{inputenc}
\usepackage{listings}

% If you use the hyperref package, please uncomment the following line
% to display URLs in blue roman font according to Springer's eBook style:
% \renewcommand\UrlFont{\color{blue}\rmfamily}

%
\title{ECMAScript 5 Specification Language}
\subtitle{Master Project in Information and Software Engineering}

%
%\titlerunning{Abbreviated paper title}
% If the paper title is too long for the running head, you can set
% an abbreviated paper title here
%
\author{Luís Miguel Alves~Loureiro\\
ist194070\\
\email{luismaloureiro@tecnico.ulisboa.pt}}
%
\authorrunning{Luís Miguel Alves Loureiro}
% First names are abbreviated in the running head.
% If there are more than two authors, 'et al.' is used.
%
\institute{Instituto Superior Técnico\\
Av. Rovisco Pais, 1\\
1049-001 Lisboa\\
Tel: +351 218 417 000\\
\email{mail@tecnico.ulisboa.pt}}
%


\begin{document}
\maketitle              % typeset the header of the contribution
%
\begin{abstract}
TODO
\keywords{ECMAScript \and Specification Language \and Programming Language \and OCaml}
\end{abstract}
%
%
%

\section{Introduction}
TODO

\subsection{Goals}
TODO

\subsection{Background}
TODO

\subsubsection{ECMAScript Standard}
TODO

\lstinputlisting[caption=Specification of the \textsc{[[}GetOwnProperty\textsc{]]} Object internal method algorithm]{GetOwnProperty.spec}

\lstinputlisting[caption=ESL code for the \textsc{[[}GetOwnProperty\textsc{]]} Object internal method algorithm]{GetOwnProperty.code}

\subsubsection{Formal Semantics}
TODO

\subsubsection{Compilers}
TODO

\section{Related Work}
TODO

\section{Evaluation}
TODO

\section{Planning}
TODO

\section{Conclusions}
TODO

%
% the environments 'definition', 'lemma', 'proposition', 'corollary',
% 'remark', and 'example' are defined in the LLNCS documentclass as well.
%

%
% ---- Bibliography ----
%
% BibTeX users should specify bibliography style 'splncs04'.
% References will then be sorted and formatted in the correct style.
%
% \bibliographystyle{splncs04}
% \bibliography{mybibliography}
%
\begin{thebibliography}{8}
\bibitem{ref_url1}
\end{thebibliography}

\end{document}
